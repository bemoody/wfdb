\documentstyle[twocolumn]{article}

\title{Installing the WFDB Software Package}
\author{}
\date{}

\begin{document}
\topmargin -0.5625in
\oddsidemargin -0.25in
\evensidemargin -0.25in
\def\columnsep{0.3125 in}
\def\textwidth{6.875 in}
\def\textheight{224 mm}
\flushbottom
% \pagestyle{empty}	% uncomment to suppress page numbering
\setcounter{page}{98}

\maketitle

\subsection*{Obtaining the WFDB Software Package}

Several versions of the WFDB Software Package, differing in their completeness,
are available.  From our web site ({\tt http://ecg.mit.edu}) you may obtain
up-to-date sources for the WFDB Software Package, as well as binary versions for
various popular operating systems.  (Currently these include x86 Linux, Sparc
Solaris, Sparc SunOS, and MS-DOS/MS-Windows.)  If you have been using earlier
versions of this software included on the first, second, or third editions of
the {\it MIT-BIH Arrhythmia Database CD-ROM}, the {\it MIT-BIH PolysomnographicDatabase CD-ROM}, the {\it European Society of Cardiology ST-T Database
CD-ROM}, or the {\it Massachusetts General Hospital/Marquette Foundation
Waveform Database CD-ROMs}, you should update your software by downloading the
newer versions available from our web site (or from one of our more recent
CD-ROMs, see below).

The complete WFDB Software Package (both sources and binaries for
Linux, Solaris, SunOS, and MS-DOS/MS-Windows) is included on the {\it
MIT-BIH Arrhythmia Database CD-ROM} (fourth edition), and also on the
{\it Software for Physiologic Databases with Samples CD-ROM},
available from MIT (see {\tt http://ecg.mit.edu/} for information
about these CD-ROMs).  Installation instructions are included in {\tt
software/README.TXT} on these CD-ROMs.

\subsection*{Choosing an installation method}

From within the {\tt software} directory of your hard disk (if you have
downloaded the WFDB Software Package from our web site or FTP server), or from
within the {\tt software} directory of your CD-ROM, follow the instructions
appropriate for your operating system.

\subsection*{If you use Linux, SunOS, or Solaris}

Examine the script {\tt install.unx} in the {\tt software} directory,
then run it (you will need root permissions to do so successfully on
most systems).  This script will identify your operating system and
install the appropriate binaries in {\tt /usr/local/bin} (other files
needed by the WFDB Software Package will be installed in other
subdirectories of {\tt /usr/local}).

If you wish to customize or recompile the WFDB Software Package, follow the
instructions in the next paragraph, for users of other Unices.

\subsection*{If you use another version of Unix}

The WFDB Software Package has been successfully compiled on dozens of Unix
variants, using both ANSI/ISO and K\&R C compilers.

Copy the source files (in the {\tt software/wfdb} directory) to a working
directory on your hard disk, then examine {\tt wfdb/Makefile} for information
about compiling the WFDB Software Package on your system.  In most cases, you
will need to change a few settings in {\tt wfdb/Makefile} to configure the
software for your system, then type {\tt make} to compile the software, and
{\tt make install} to install it.  You will need root permissions to install
the software in the standard locations.  

If you would like to use {\sf WAVE} on a UNIX system other than Linux,
Solaris, or SunOS, you will need to port XView to your system first
(or purchase a commercial port if one is available).  Sources for
XView are supplied on our CD-ROMs, and are also available from our
web/FTP server ({\tt ecg.mit.edu}) as well as from {\tt
sunsite.unc.edu}, {\tt tsx-11.mit.edu}, and their mirrors.  \emph{We
cannot offer assistance in porting XView; if you wish to try this, you
are on your own.}  If you successfully port the {\tt cmdtool} terminal
emulator application included in the XView sources, we will assist you
in porting {\sf WAVE} (this is much simpler than the XView port).

\subsection*{If you use MS-DOS or MS-Windows}

Run {\tt install.exe} from the {\tt software} directory.  Follow the
on-screen directions to choose the installation directories, to set up
your WFDB path, and to calibrate your display.  Detailed instructions
may be found in {\tt software/MSDOS.TXT}.

The source files within the {\tt wfdb} directory are in UNIX text format
(newlines are marked by ASCII line feed characters only).  Most MS-DOS C and
C++ compilers and text editors can process UNIX-format text files without
difficulty; if you wish to convert these files to MS-DOS native text file
format, however, first install {\tt msdos/bin/u2d.exe} into a directory in your
{\tt PATH}, copy the UNIX-format files to a writable directory, change to that
directory, and type `{\tt u2d *.*}'.   (You may name individual files to be
converted if you prefer. `{\tt u2d}' does not modify binary files or files
that are already in MS-DOS text format.) 
 
If you wish, you can compile the sources using Microsoft or Borland C or C++
compilers without modification; see the files named {\tt makefile.dos} in the
subdirectories of `wfdb' for details.  If you have Microsoft or Turbo C or C++,
and a Microstar Laboratories DAP 1200- or 2400-series analog interface board,
you can recompile {\tt sample} (a program for creating database records from
analog signals, and for replaying them in analog form).  To do so successfully,
you must first have installed the appropriate Microstar {\tt \#include} files
and DAP interface library for use with your C compiler.  Specifically, files
{\tt c\_lib.c}, {\tt clock.h}, and {\tt ioutil.h} must be installed in your
{\tt include} directory, and file {\tt cdapl.lib} must be installed in a
directory in which libraries are found by your linker.

\subsection*{If you use a Macintosh or another system}

The WFDB Software Package is written in highly portable C, and (with
the exception of a few MS-DOS or UNIX-specific display or
data-acquisition programs) should be easy to compile with any K\&R or
ANSI C compiler.  The UNIX and MS-DOS {\tt make} description files
({\tt Makefile} and {\tt makefile.dos} in {\tt software/wfdb} and in
each of its subdirectories) should get you started.

Notes for Macintosh users can be found in {\tt software/MAC.TXT}, and
in {\tt software/wfdb/MAC.TXT}.  These include detailed instructions
for compiling the WFDB library using Symantec's Think C.  Since most
of the applications are command-line oriented, they will require minor
modifications to run under the Mac OS.

\section*{Your comments are requested}

I would greatly appreciate a report of any problems you encounter in installing
or using this software, if possible by e-mail to {\tt george@mit.edu},
or to:
\begin{center}
George B. Moody\\MIT Room E25-505A\\Cambridge, MA 02139 USA
\end{center}
\end{document}
