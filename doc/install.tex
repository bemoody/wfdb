\documentstyle[twocolumn]{article}

\title{Installing the WFDB Software Package}
\author{George B. Moody\\
Harvard-MIT Division of Health Sciences and Technology, Cambridge, MA, USA}
\date{}

\begin{document}
\topmargin -0.5625in
\oddsidemargin -0.25in
\evensidemargin -0.25in
\def\columnsep{0.3125 in}
\def\textwidth{6.875 in}
\def\textheight{224 mm}
\flushbottom
% \pagestyle{empty}	% uncomment to suppress page numbering
\setcounter{page}{105}

\maketitle

\section*{Obtaining the WFDB Software Package}

Several versions of the WFDB Software Package, differing in their completeness,
are available.  From PhysioNet ({\tt http://www.physionet.org/}) you may obtain
up-to-date sources for the WFDB Software Package, including WAVE and WVIEW, as
well as binary versions for various popular operating systems.  (Currently
these include x86 Linux, Sparc Solaris, Sparc SunOS, and MS-DOS/MS-Windows.)
If you have been using earlier versions of this software included on the first,
second, or third editions of the {\it MIT-BIH Arrhythmia Database CD-ROM}, the
{\it MIT-BIH PolysomnographicDatabase CD-ROM}, the {\it European Society of
Cardiology ST-T Database CD-ROM}, or the {\it Massachusetts General
Hospital/Marquette Foundation Waveform Database CD-ROMs}, you should update
your software by downloading the newer versions available from PhysioNet (or
from one of our more recent CD-ROMs, see below).

The most recent stable version of the package may be obtained by downloading
either {\tt http://www.\-physio\-net.\-org/\-physio\-tools/\-wfdb.tar.gz}
(sources and documentation) or {\tt
http://www.\-physio\-net.\-org/\-physio\-tools/\-wfdb-no-docs.tar.gz}
(sources only).
There may also be a development version available at either {\tt
http://www.\-physio\-net.\-org/\-physio\-tools/\-beta/\-wfdb.tar.gz}
(sources and documentation) or {\tt
http://www.\-physio\-net.\-org/\-physio\-tools/\-beta/\-wfdb-no-docs.tar.gz}
(sources only).
These files are gzip-compressed UNIX tar archives, which can be decompressed
and unpacked using {\tt gzip} and {\tt tar} under any version of UNIX using
a command such as
\begin{verbatim}
gzip -d <wfdb.tar.gz | tar xfv -
\end{verbatim}
or (with modern versions of {\tt tar})
\begin{verbatim}
tar xfvz wfdb.tar.gz
\end{verbatim}
This creates a source tree beginning with a directory named {\tt wfdb}.  {\tt
gzip} and {\tt tar} are also freely available for MS-DOS, MS-Windows, OS/2,
and many other operating systems.  The popular WinZip utility for MS-Windows
can also decompress and unpack these files, but be sure to restore their
original names if your browser has renamed them (WinZip can only recognize the
format if the file name ends in ``{\tt .tar.gz}'', and some browsers will
change the first ``.'' to ``\_'' in the name of the downloaded file).

Alternatively, you may download individual files from the on-line WFDB source
tree ({\tt http://www.\-physio\-net.\-org/\-physio\-tools/\-wfdb/} for
the stable version,
or ({\tt http://www.\-physio\-net\-.org/\-physio\-tools/\-beta/\-wfdb/} for
the development version).

The complete WFDB Software Package (both sources and binaries for
Linux, Solaris, SunOS, and MS-DOS/MS-Windows) is also included on
the {\it MIT-BIH Arrhythmia Database CD-ROM} (fourth edition) and on the
{\it Software for Physiologic Databases with Samples CD-ROM}, both
available from MIT (see {\tt http://ecg.mit.edu/} for information
about these CD-ROMs).  If you wish to compile the WFDB sources from one of
these CD-ROMs, begin by copying the entire contents of the {\tt wfdb} source
tree from the CD-ROM to a writable hard drive.

\section*{Choosing an installation method}

From within the {\tt wfdb} directory of your hard disk (if you have
downloaded the WFDB Software Package from PhysioNet), or from
within the {\tt software} directory of your CD-ROM, follow the instructions
appropriate for your operating system.

\subsection*{If you use Linux, SunOS, or Solaris}

Examine {\tt wfdb/Makefile} for information about compiling the WFDB Software
Package on your system.  In most cases, you will need to change a few settings
in {\tt wfdb/Makefile} to configure the software for your system, then type
{\tt make} to compile and install the software.  You will need root permissions
to install the software in the standard locations.

\subsection*{If you use another version of Unix}

The WFDB Software Package has been successfully compiled on dozens of Unix
variants, using both ANSI/ISO and K\&R C compilers.

If you would like to use {\sf WAVE} on a UNIX system other than Linux,
Solaris, or SunOS, you will need to port XView to your system first
(or purchase a commercial port if one is available).  Sources for
XView are supplied on our CD-ROMs, and are also available from PhysioNet
({\tt www.physionet.org}) as well as from {\tt
metalab.unc.edu}, {\tt tsx-11.mit.edu}, and their mirrors.  \emph{We
cannot offer assistance in porting XView; if you wish to try this, you
are on your own.}  If you successfully port the {\tt cmdtool} terminal
emulator application included in the XView sources, we will assist you
in porting {\sf WAVE} (this is much simpler than the XView port).

If you are not planning to port {\sf WAVE}, edit {\tt wfdb/Makefile} to
remove any lines that begin with ``{\tt cd wave;}''.

Make any other necessary changes in {\tt wfdb/Makefile} for your
system, then type {\tt make} to compile and install the WFDB Software
Package.  You will need root permissions to install the software in the
standard locations.


\subsection*{If you use MS-DOS or MS-Windows}

%% This is out-of-date.  For now, look at the files named 'makefile.dos'
%% in the subdirectories of 'wfdb'.
% Run {\tt install.exe} from the {\tt software} directory.  Follow the
% on-screen directions to choose the installation directories, to set up
% your WFDB path, and to calibrate your display.  Detailed instructions
% may be found in {\tt software/MSDOS.TXT}.

The source files within the {\tt wfdb} directory are in UNIX text format
(newlines are marked by ASCII line feed characters only).  Most MS-DOS C and
C++ compilers and text editors can process UNIX-format text files without
difficulty; if you wish to convert these files to MS-DOS native text file
format, however, first install {\tt u2d} (available from
{tt http://www\-.physio\-net.org/\-physio\-tools/\-bi\-naries/\-msdos/\-bin/u2d.exe}) into a directory in your
{\tt PATH}, copy the UNIX-format files to a writable directory, change to that
directory, and type `{\tt u2d *.*}'.   (You may name individual files to be
converted if you prefer. `{\tt u2d}' does not modify binary files or files
that are already in MS-DOS text format.) 
 
If you wish, you can compile the sources using Microsoft or Borland C or C++
compilers without modification; see the files named {\tt makefile.dos} in the
subdirectories of `wfdb' for details.  If you have Microsoft or Turbo C or C++,
and a Microstar Laboratories DAP 1200- or 2400-series analog interface board,
you can recompile {\tt sample} (a program for creating database records from
analog signals, and for replaying them in analog form).  To do so successfully,
you must first have installed the appropriate Microstar {\tt \#include} files
and DAP interface library for use with your C compiler.  Specifically, files
{\tt c\_lib.c}, {\tt clock.h}, and {\tt ioutil.h} must be installed in your
{\tt include} directory, and file {\tt cdapl.lib} must be installed in a
directory in which libraries are found by your linker.

\subsection*{If you use a Macintosh or another system}

The WFDB Software Package is written in highly portable C, and (with
the exception of a few MS-DOS or UNIX-specific display or
data-acquisition programs) should be easy to compile with any K\&R or
ANSI C compiler.  The UNIX and MS-DOS {\tt make} description files
({\tt Makefile} and {\tt makefile.dos} in {\tt software/wfdb} and in
each of its subdirectories) should get you started.

%% This is also out-of-date.
% Notes for Macintosh users can be found in {\tt software/MAC.TXT}, and
% in {\tt software/wfdb/MAC.TXT}.  These include detailed instructions
% for compiling the WFDB library using Symantec's Think C.  Since most
% of the applications are command-line oriented, they will require minor
% modifications to run under the Mac OS.

\section*{Your comments are requested}

I would greatly appreciate a report of any problems you encounter in installing
or using this software, if possible by e-mail to {\tt george@mit.edu},
or to:
\begin{center}
George B. Moody\\MIT Room E25-505A\\Cambridge, MA 02139 USA
\end{center}
\end{document}
